\documentclass[addpoints, 12pt]{exam}
\usepackage[top=0.75in, bottom=0.75in, left=0.75in, right=0.75in]{geometry}

\usepackage{multirow}
\usepackage{graphicx}
\usepackage{caption}
\usepackage{subcaption}
\usepackage{epstopdf}
\usepackage{footnote}
\usepackage{fmtcount}
\usepackage{array}
\usepackage{setspace}
\usepackage{enumerate}
\usepackage{verbatim}
\usepackage{url}
\usepackage{amsmath}

\newcommand{\fillanswer}[2][\fill]{%
  \unskip\ \lhrulefill{#1}%
  \ifteacher\makebox[0pt]{#2}\fi
  \lhrulefill{#1}\ \ignorespaces}
\newcommand{\lhrulefill}[1]{%
  \leavevmode
  \leaders\hrule height -.3ex depth \dimexpr .3ex+.4pt\relax % define the leader
  \hskip\glueexpr#1/2\relax\relax % how much it should extend
  \kern0pt
}


\title{MGMTMSA 408 -- Operations Analytics}
\author{Homework 4 -- Revenue Management and Assortment Optimization (Answer Sheet)}
\date{Due June 3, 2020 at 1pm PST}
\begin{document}

\maketitle 
\noindent \textbf{Name:} \enspace \fillanswer{Tigran Margarian} 
\textbf{UID:} \enspace \fillanswer{105451572} \\

\noindent \textbf{Note}: Please submit your Python code on CCLE in order to receive full credit. 

\section{Cloud Computing}

\subsection*{Part 1: Capacity control formulation}

\begin{enumerate}[a)]
\item $\phantom{=}$ \vspace{-2em}
	\fbox{\begin{minipage}[2\height]{\dimexpr\textwidth-11\fboxsep\fboxrule\relax}
 	In this setting our goal is basically to decide on $x_i$ to increase the overall
	revenue. Let's note the price of each instance as $r_i$. Then our objective 
	function which we have to maximize would look like:
	
	$$ \sum_{i = 1}^{9} r_{i}x_{i} $$
	
	\end{minipage}} \\\\
\item $\phantom{=}$ \vspace{-2em}
	\fbox{\begin{minipage}[2\height]{\dimexpr\textwidth-11\fboxsep\fboxrule\relax}
	Overall there is 1024 GB of memory available. In that case the memory constraint 
	would look like: 

	$$8x_{1} + 16x_{2} + 32x_{3} + 32x_{4} + 64x_{5} + 128x_{6} + 16x_{7} + 32x_{8} + 
	64x_{9} \leq 1024 $$

	\end{minipage}} \\\\
\item $\phantom{=}$ \vspace{-2em}
	\fbox{\begin{minipage}[2\height]{\dimexpr\textwidth-11\fboxsep\fboxrule\relax}
	Assuming that instance requests follow the Poisson distribution with $\lambda$ 
	being represented by Rate (\# / day) the expected number of requests for 
	instance 5 over 5 day interval would be equal to $5\lambda_{5} = 5 \times 2.6 =
	13$. \medskip

	The corresponding constraint on the request numbers would look like: 
	$$x_{5} \leq 13$$
	\end{minipage}} \\\\

\item $\phantom{=}$ \vspace{-2em}
	\fbox{\begin{minipage}[2\height]{\dimexpr\textwidth-11\fboxsep\fboxrule\relax}
	The LP formulation for T = 5 capacity control problem: \medskip

	$$maximize \sum_{i = 1}^{9} r_{i}x_{i} $$
	
	subject to: \medskip

	$8x_{1} + 16x_{2} + 32x_{3} + 32x_{4} + 64x_{5} + 128x_{6} + 16x_{7} + 32x_{8} + 
	64x_{9} \leq 1024 $ (memory constraint) \\
	
	$16x_{1} + 32x_{2} + 64x_{3} + 8x_{4} + 16x_{5} + 32x_{6} + 16x_{7} + 32x_{8} + 
	64x_{9} \leq 512$ (CPU constraint) \\
	
	$x_1 + x_2 + x_3 + x_4 + x_5 + x_6 + 2x_7 + 6x_8 + 8x_9 \leq 64$ (GPU constraint)\\
	
	$x_1 \leq 25$; $x_2 \leq 25$; $x_3 \leq 9$; $x_4 \leq 15$; $x_5 \leq 13$; $x_6 \leq 5$;
	$x_7 \leq 4$; $x_8 \leq 2$; $x_9 \leq 1.5$\\ 
	(demand constraints) \\

	$x_i \geq 0,~\forall i \in \{1,...,9\}$ (non-negativity constraints)
	\end{minipage}} \\\\
\end{enumerate}
 
 \subsection*{Part 2: Solving the capacity control problem in Python/Gurobi}
 \begin{enumerate}[a)]
\item $\phantom{=}$ \vspace{-2em}
	\fbox{\begin{minipage}[2\height]{\dimexpr\textwidth-11\fboxsep\fboxrule\relax}
	Optimal objective value (revenue) equals \$1039.43 
	\end{minipage}} \\\\
\item $\phantom{=}$ \vspace{-2em}
	\fbox{\begin{minipage}[2\height]{\dimexpr\textwidth-11\fboxsep\fboxrule\relax}
	There according to the optimizer output there are 6.29 C1 instance requests accepted. 
	\end{minipage}} \\\\
\item $\phantom{=}$ \vspace{-2em}
	\fbox{\begin{minipage}[2\height]{\dimexpr\textwidth-11\fboxsep\fboxrule\relax}
	It doesn't make sense to add more GPU capacity because according to the optimized 
	solution we still have 17.29 GBs of GPU memory unused.
	\end{minipage}} \\\\
\item $\phantom{=}$ \vspace{-2em}
	\fbox{\begin{minipage}[2\height]{\dimexpr\textwidth-11\fboxsep\fboxrule\relax}
	Using the shadow prices of CPU and memory we can calculate that the considered
	server would bring us \$9 of additional revenue 
	\end{minipage}} \\\\
\end{enumerate}

 \subsection*{Part 3: Simulating current practice}
 \begin{enumerate}[a)]
\item $\phantom{=}$ \vspace{-2em}
	\fbox{\begin{minipage}[2\height]{\dimexpr\textwidth-11\fboxsep\fboxrule\relax}
	The average number of arrivals of type C1 instance in the set of simulated 
	sequences is 26.63
	\end{minipage}} \\\\
\item $\phantom{=}$ \vspace{-2em}
	\fbox{\begin{minipage}[2\height]{\dimexpr\textwidth-11\fboxsep\fboxrule\relax}
	The average number of arrivals of all types of instances over the set of 
	simulated sequences is 101. \medskip

	We know that for random variables obeying the Poisson distribution X and Y
	the random variable Z = X + Y would have a Poisson distribution as well
	with the expected value E(Z) = E(X) + E(Y). Therefore the obtained number
	makes perfect sense because it's almost the same as the sum for forecasts
	for all the instance types (99.5) and that is expected by theory.
	\end{minipage}} \\\\
\item $\phantom{=}$ \vspace{-2em}
	\fbox{\begin{minipage}[2\height]{\dimexpr\textwidth-11\fboxsep\fboxrule\relax}
	The average revenue generated by the myopic strategy is \$528.28
	\end{minipage}} \\\\
\item $\phantom{=}$ \vspace{-2em}
	\fbox{\begin{minipage}[2\height]{\dimexpr\textwidth-11\fboxsep\fboxrule\relax}
	The average remaining capacities are: 0.24 CPUs, 371.52 GBs of memory and 
	37.42 GBs of GPU memory. It seems like this following this strategy we 
	quickly run out of CPUs and are getting left with plenty of memory and GPU 
	capacity.
	\end{minipage}} \\\\
\end{enumerate}

 \subsection*{Part 4: A bid-price control policy}
 \begin{enumerate}[a)]
\item $\phantom{=}$ \vspace{-2em}
	\fbox{\begin{minipage}[2\height]{\dimexpr\textwidth-11\fboxsep\fboxrule\relax}
	For an instance of type 5 (M2) requiring 16 CPUs, 64 GBs of memory and 1 GB of GPU
	capacity the opportunity cost in terms of shadow prices would look like:

	$$Cost~of~M2~instance~acceptance = 16\pi_1 + 64\pi_2 + \pi_3$$
	\end{minipage}} \\\\
\item $\phantom{=}$ \vspace{-2em}
	\fbox{\begin{minipage}[2\height]{\dimexpr\textwidth-11\fboxsep\fboxrule\relax}
	The expected number of requests of instance $i$ with Poisson arrival rate of $\lambda_i$
	from time $t$ to $T$ is $(T-t+1)\times\lambda_i$
	\end{minipage}} \\\\
\item $\phantom{=}$ \vspace{-2em}
	\fbox{\begin{minipage}[2\height]{\dimexpr\textwidth-11\fboxsep\fboxrule\relax}
	The average revenue generated by the bid-price control  strategy is \$925.59 
	which is much closer to the \textit{ideal} result that the myopic strategy revenue.
	\end{minipage}} \\\\
\item $\phantom{=}$ \vspace{-2em}
	\fbox{\begin{minipage}[2\height]{\dimexpr\textwidth-11\fboxsep\fboxrule\relax}
	The average remaining capacities are: 27.2 CPUs, 4.88 GBs of memory and 
	20.62 GBs of GPU memory. The distribution of capacity is much more even 
	with bid-price control policy.
	\end{minipage}} \\\\
\end{enumerate}

 
\pagebreak 

\section{Designing a Sushi Menu}

\subsection*{Part 1: Understanding the data}

\begin{enumerate}[a)]
\item $\phantom{=}$ \vspace{-2em}
	\fbox{\begin{minipage}[2\height]{\dimexpr\textwidth-11\fboxsep\fboxrule\relax}
	The top five sushis preferred by customer 1 are:
	\begin{itemize}
		\item negi-toro \textit{(category 1)} - utility = 2.911
		\item ika \textit{(category 5)} - utility = 2.808
		\item maguro \textit{(category 1)} - utility = 2.806
		\item samon \textit{(category 1)} - utility = 2.772
		\item inari \textit{(category 11)} - utility = 2.751
	\end{itemize}
	Seems like customer 1 is a big fan of sushis belonging to the category 1 (tuna/salmon sushis).
	\end{minipage}} \\\\
\item $\phantom{=}$ \vspace{-2em}
	\fbox{\begin{minipage}[2\height]{\dimexpr\textwidth-11\fboxsep\fboxrule\relax}
	The worst five sushis for customer 2 are:
	\begin{itemize}
		\item karasumi \textit{(category 7)} - utility = -0.404
		\item komochi-konbu \textit{(category 7)} - utility = -0.317
		\item kyabia \textit{(category 7)} - utility = -0.258
		\item awabi \textit{(category 4)} - utility = -0.257
		\item kazunoko \textit{(category 7)} - utility = -0.231
	\end{itemize}
	We can clearly see that customer 2 hates category 7 sushis (almost all of them include roe or caviar).
	\end{minipage}} \\\\
\item $\phantom{=}$ \vspace{-2em}
	\fbox{\begin{minipage}[2\height]{\dimexpr\textwidth-11\fboxsep\fboxrule\relax}
	The top five sushis by average ranking (with rank = 1 as best and rank = 100 as worst) are:
	\begin{itemize}
		\item negi-toro \textit{(category 1)} - avg. ranking = 7.876
		\item maguro \textit{(category 1)} - avg. ranking = 10.050 
		\item kurumaebi \textit{(category 6)} - avg. ranking = 10.084
		\item negi-toro-maki \textit{(category 1)} - avg. ranking = 10.214
		\item ebi \textit{(category 6)} - avg. ranking = 11.932
	\end{itemize}
	We can see that the top-5 sushis for all the customers surveyed are belonging either to category 1 or 6.
	Category 1 us characterized by including tuna and category 6 sushis mostly include shrimp.  
	\end{minipage}} \\\\
\item $\phantom{=}$ \vspace{-2em}
	\fbox{\begin{minipage}[2\height]{\dimexpr\textwidth-11\fboxsep\fboxrule\relax}
	The sushi with the worst average ranking is \textbf{kyabia} with 
	average ranking value of 90.648. 
	\end{minipage}} \\\\
\item $\phantom{=}$ \vspace{-2em}
	\fbox{\begin{minipage}[2\height]{\dimexpr\textwidth-11\fboxsep\fboxrule\relax}
	The most controversial sushi based on ranking standard deviation is \textbf{awabi} 
	with a ranking SD of 27.439 and average ranking value of 62.592. 
	\end{minipage}} \\\\
\end{enumerate}

\subsection*{Part 2: Common-sense solutions}

\begin{enumerate}[a)]
\item $\phantom{=}$ \vspace{-2em}
	\fbox{\begin{minipage}[2\height]{\dimexpr\textwidth-11\fboxsep\fboxrule\relax}
	The expected revenue from offering all the sushis is \$21.51
	\end{minipage}} \\\\
\item $\phantom{=}$ \vspace{-2em}
	\fbox{\begin{minipage}[2\height]{\dimexpr\textwidth-11\fboxsep\fboxrule\relax}
	The ten most expensive sushis are: \textbf{toro, awabi, tarabagani, akagai, umi, kurumaebi, 
	chu-toro, suzuki, hirame, botanebi}. \medskip 

	The expected revenue with this set is equal to \$25.64 
	\end{minipage}} \\\\
\item $\phantom{=}$ \vspace{-2em}
	\fbox{\begin{minipage}[2\height]{\dimexpr\textwidth-11\fboxsep\fboxrule\relax}
	The expected revenue from offering the customers' favorite sushis is \$21.51. \medskip

	Interestingly that is the same result we obtained from offering all the sushis. I believe
	it happens because from the whole variety of sushis we offer in part a), customers choose one bringing
	the highest utility for them. In part c) we explicitly arrange the menu in such way to include the 
	favorite sushi of each customer. Therefore, in both cases customers end up with the same choice 
	delivering the same revenue to the company.
	\end{minipage}} \\\\
\item $\phantom{=}$ \vspace{-2em}
	\fbox{\begin{minipage}[2\height]{\dimexpr\textwidth-11\fboxsep\fboxrule\relax}
	There are several reasons why offering all sushis would be suboptimal strategy. First of all, 
	clearly there are some sushis that would never be ordered but their offering will have an impact
	on the budget of the restaurant. Second, by offering everything we provide our customers an 
	opportunity to choose the highest utility option which may be actually not very profitable
	for the restaurant. If in turn we could find a set what would still provide an option with 
	a utility higher than 3 (no option) and additionally boost the expected profit maximally - that 
	would provide a more optimal solution.
	\end{minipage}} \\\\
\item $\phantom{=}$ \vspace{-2em}
	\fbox{\begin{minipage}[2\height]{\dimexpr\textwidth-11\fboxsep\fboxrule\relax}
	The approach of offering the most expensive and revenue driving sushis may be suboptimal because
	usually there is a negative correlation between utility of a good and its price. Customers don't 
	quite enjoy paying much and therefore leaving only expensive choices my force many customers to 
	leave the restaurant without an order what will in turn decrease the potential revenue.
	\end{minipage}} \\\\
\end{enumerate}

\subsection*{Part 3: An integer optimization model}

\begin{enumerate}[a)]
\item $\phantom{=}$ \vspace{-2em}
	\fbox{\begin{minipage}[2\height]{\dimexpr\textwidth-11\fboxsep\fboxrule\relax}
	In this model $y_{k,i}$ is a binary variable which equals to 1 if k-th customer chooses 
	the i-th sushi and zero otherwise. $x_i$ is binary variable indicating whether we include 
	the i-th sushi in the menu. $r_i$ is the price of i-th sushi and finally $u_{k,j}$ is the 
	utility of j-th sushi for the k-th customer.
	Let's go over the considered constraints:
	\begin{itemize}
		\item The 1b) constraint means that each customer can choose only one option  
			(order one type of sushi or walk away).
		\item The 1c) constraint means that the utility of the option chosen by each 
			customer should be greater or equal than the utility of other options
			(that said each customer's choice has the highest utility available).
		\item The 1d) constraint means that the utility of the option chosen by each 
			customer should be greater or equal then the utility of non-purchase option 
			(that said each customer's choice shouldn't have a lower utility than the 
			non-purchase option).
		\item The 1e) constraint means that it's impossible for the customer to choose 
			an option not offered by the restaurant.
	\end{itemize}
	\end{minipage}} \\\\
\item $\phantom{=}$ \vspace{-2em}
	\fbox{\begin{minipage}[2\height]{\dimexpr\textwidth-11\fboxsep\fboxrule\relax}
	The optimized revenue from LP relaxation equals to \$30.59.
	\end{minipage}} \\\\
\item $\phantom{=}$ \vspace{-2em}
	\fbox{\begin{minipage}[2\height]{\dimexpr\textwidth-11\fboxsep\fboxrule\relax}
	That is impossible because even the linear LP relaxation gave the optimal revenue 
	of \$30.59 per customer. That means that the integer LP output cannot be greater 
	and therefore a set of menu delivering \$32 of per-customer revenue cannot exist.
	\end{minipage}} \\\\
\item $\phantom{=}$ \vspace{-2em}
	\fbox{\begin{minipage}[2\height]{\dimexpr\textwidth-11\fboxsep\fboxrule\relax}
	The optimized revenue from integer LP equals to \$26.24 what is \$4.73 more than 
	the result of common-sense strategy of offering all sushis (or only customers' favorites) and 
	\$0.60 more than the result of expensive sushi offering. We should also keep in mind that this 
	is the per-customer level revenue which means that 60 additional cents from customer can convert
	into much higher number depending on how much clients the restaurant can serve.
	\end{minipage}} \\\\
\item $\phantom{=}$ \vspace{-2em}
	\fbox{\begin{minipage}[2\height]{\dimexpr\textwidth-11\fboxsep\fboxrule\relax}
	The optimal set of sushis now includes 7 options: \textbf{toro, awabi, saba, tarabagani,
	mentaiko-maki, ika-nattou, tobiuo}.
	\end{minipage}} \\\\
\end{enumerate}
\pagebreak

\subsection*{Part 4: A constrained model}

\begin{enumerate}[a)]
\item $\phantom{=}$ \vspace{-2em}
	\fbox{\begin{minipage}[2\height]{\dimexpr\textwidth-11\fboxsep\fboxrule\relax}
	First we should reshape the category data into 2 dimensional matrix $c_{100,12}$ of 
	binary variables where $c_{i,c} = 1$ if the \textit{i}-th sushi belongs to the category \textit{c}
	and zero otherwise. With that we can add the necessary constraint which will look like: 

	$$\sum_{i=1}^{n} x_{i} c_{i,c} \geq 1~~ \forall~c \in \{1,2,...,12\}$$

	This condition will force the optimizer to keep at least one sushi of each category.
	\end{minipage}} \\\\
\item $\phantom{=}$ \vspace{-2em}
\fbox{\begin{minipage}[2\height]{\dimexpr\textwidth-11\fboxsep\fboxrule\relax}
	The optimized revenue from constrained integer LP equals to \$25.88.
	\end{minipage}} \\\\
\item $\phantom{=}$ \vspace{-2em}
	\fbox{\begin{minipage}[2\height]{\dimexpr\textwidth-11\fboxsep\fboxrule\relax}
	The optimal set of sushis now includes 19 options: \textbf{tamago, toro, awabi,
	shako, saba, unagi, geso, tarabagani, ankimo, gyusashi, fugu, okura, ika-nattou,
	kaiware, kue, kamo, mamakari, kyabia, kaki.}
	\end{minipage}} \\\\
\end{enumerate}

\subsection*{Part 5: Next steps}
\begin{enumerate}[a)]
\item $\phantom{=}$ \vspace{-2em}
\fbox{\begin{minipage}[2\height]{\dimexpr\textwidth-11\fboxsep\fboxrule\relax}
	Some limitations of this model I can think about are:
	\begin{itemize}
		\item The fact that we assume that each customer is allowed to purchase only one 
			single sushi. It may easily be the case in real conditions that some customers 
			would choose more than one sushi. 
		\item The equal probability of each customer. The created model assumes that there is 
			an equal probability of each customer to show up what may turn out to be wrong.
		\item Finally, we assume that the survey conducted with 500 customers should be descriptive
			for all future customers and that also may turn out to be wrong.
	\end{itemize} 
	\end{minipage}} \\\\
\item $\phantom{=}$ \vspace{-2em}
\fbox{\begin{minipage}[2\height]{\dimexpr\textwidth-11\fboxsep\fboxrule\relax}
	One approach which may help to reduce the number of customers considered may be clustering.
	We can implement clustering algorithms like K-Means or Factor Analysis to group customers 
	with similar preferences. That may in turn significantly simplify calculations for the optimizer.
	\end{minipage}} \\\\
\end{enumerate}


\end{document}
